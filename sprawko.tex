\documentclass[12pt,a4paper]{article}
\usepackage[margin=0.5in]{geometry}

\usepackage{polski}
\usepackage[utf8]{inputenc}

\usepackage{graphicx}
\usepackage{float}
\usepackage{setspace} %texlive-latex-recommended
\usepackage{enumitem}

\usepackage{tabularx}

\newcommand{\imgz}{0.44\linewidth}
\newcommand{\numerzajec}{2}
\newcommand{\tematzajec}{Problem śpiącego fryzjera.}
\newcommand{\datazajec}{8 czerwca 2017}


\begin{document}
	\thispagestyle{empty}
	\begin{center}
		\vspace*{1.6cm}
		\includegraphics[width=0.55\linewidth]{pblogo.png}	\\
		\vspace{0.5cm}
		\large
		\textsf{\textbf{Sprawozdanie z pracowni specjalistycznej}} \\
		\vspace{0.5cm}
		\textsf{\textbf{\textit{Systemy operacyjne}}}	\\
		\vspace{1cm}
		\textsf{Projekt numer: \textbf{\numerzajec}}	\\
		\vspace{0.5cm}
		\textsf{Temat: \textbf{\tematzajec}}
	\end{center}

	\vspace{2cm}
	\begin{tabular}{rl}
        Wykonujący ćwiczenie: &\textbf{Karol Budlewski} \\
                              & \textbf{Karol Budlewski} \\
	\end{tabular}

	\vspace{3.5cm}	

	\begin{minipage}{0.45\linewidth}
		\large
		\begin{spacing}{1.5}
		Studia dzienne \\
		Kierunek: Informatyka \\
		Semestr: IV \\
		\end{spacing}
	\end{minipage}
	\begin{minipage}[t]{0.5\linewidth}
		\large
		\begin{spacing}{1.2}
		Grupa zajęciowa: PS4
		\end{spacing}
	\end{minipage}
	
	Prowadzący ćwiczenie: \textbf{mgr inż. Daniel Reska} \\
	
	\begin{flushright}
		\begin{minipage}[t]{0.3\linewidth}
			\centering
			................. \\
			\small OCENA
		\end{minipage}
	\end{flushright}
	
	\begin{minipage}[t]{0.4\linewidth}
		\centering
		Data \\
		\small \datazajec
	\end{minipage}
	
	\begin{flushright}
		\begin{minipage}[t]{0.5\linewidth}
			\centering
			............................................. \\
			\small \textsf{Data i podpis prowadzącego}
		\end{minipage}
	\end{flushright}
	\pagebreak
	%%%%%%%%%%%%%%%%%%%%%%%%%%%%%%%%%%%%%%%%%%%%%%%%%

	\tableofcontents

	\pagebreak
	%%%%%%%%%%%%%%%%%%%%%%%%%%%%%%%%%%%%%%%%%%%%%%%%%

	
	\section{Wstęp}
	Celem projektu było rozwiązanie problemu fryzjera realizowanego na wątkach oraz 
	synchronizowanego semaforami, mutexami i zmiennymi warunkowymi.\\
	Program zrealizowany jest w wariantach:
	\begin{enumerate}[label=\alph*)]
		\item z użyciem wyłącznie semaforów i mutexów.
		\item z użyciem zmiennych warunkowych.
	\end{enumerate}
	Oczekiwana liczba punktów to \texttt{34}.
	
	\section{Instalacja}
	Program jest instalowany z linii komend. Należy przejść do katalogu z plikiem 
	\texttt{Makefile} i uruchomić komendę \texttt{make}. Po tym mamy już skompilowany 
	program, który mieści się w katalogu \texttt{build}.
	
	\section{Sposób uruchamiania}
	Program jest uruchamiany z linii komend i może przyjmować następujące parametry:
	\begin{enumerate}[label=\alph*)]
		\item \texttt{-t}  - określa czas w \texttt{mikrosekundach} z jakim będą 
				się pojawiać nowe wątki \texttt{Klient}
		\item \texttt{-ts} - określa czas w \texttt{sekundach} z jakim będą 
				się pojawiać nowe wątki \texttt{Klient}
		\item \texttt{-tms} - określa czas w \texttt{milisekundach} z jakim będą 
				się pojawiać nowe wątki \texttt{Klient}
		\item \texttt{-size} - określa wielkość poczekalni
		\item \texttt{-debug} - uruchamia tryb \texttt{debug}, wypisujący całą 
				kolejkę klientów czekających oraz tych którzy nie dostali się do 
				gabinetu 
	\end{enumerate}
	Do uruchomienia programu nie wymagane są żadne parametry. Program uruchomi 
	się wówczas z ustawieniami domyślnymi, czyli czas ustawiony na 75 ms, rozmiar 
	poczekalni na 10 oraz bez debuga. Parametry \texttt{-t}, \texttt{-ts}, \texttt{-tms} 
	wpływają na siebie i zawsze będzie brany ostatni pod uwagę oraz jeżeli zostanie 
	popełniony błąd w którym kolwiek z opcji program nie zostanie uruchomiony.
	\subsubsection*{Przykłady uruchamiania programu}
	\begin{itemize}
		\item ./build/fryzjer -t   150
		\item ./build/fryzjer -ts  150
		\item ./build/fryzjer -tms 150
		\item ./build/fryzjer -ts  150 -debug
		\item ./build/fryzjer -tms 150 -size 10
		\item ./build/fryzjer -t   150 -size 10 -debug
	\end{itemize}
 	
 	\section{Testowanie}
 	
 	\section{Zaimplementowane funkcje}
 	\subsection{Wersja z użyciem zmiennych warunkowych}

 	

\end{document}
