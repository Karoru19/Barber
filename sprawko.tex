\documentclass[12pt,a4paper]{article}
\usepackage[margin=0.5in]{geometry}

\usepackage{polski}
\usepackage[utf8]{inputenc}

\usepackage{graphicx}
\usepackage{float}
\usepackage{setspace} %texlive-latex-recommended
\usepackage{enumitem}
\usepackage{listings}
\usepackage{xcolor}
\usepackage{color}
\usepackage{tabularx}
 
\definecolor{codegreen}{rgb}{0,0.6,0}
\definecolor{codegray}{rgb}{0.5,0.5,0.5}
\definecolor{codepurple}{rgb}{0.58,0,0.82}
\definecolor{backcolour}{rgb}{0.95,0.95,0.92}

\newcommand{\imgz}{0.44\linewidth}
\newcommand{\numerzajec}{2}
\newcommand{\tematzajec}{Problem śpiącego fryzjera.}
\newcommand{\datazajec}{8 czerwca 2017}

\lstdefinestyle{customc}{
  belowcaptionskip=1\baselineskip,
  breaklines=true,
  xleftmargin=\parindent,
  language=C,
   backgroundcolor=\color{backcolour},   
    commentstyle=\color{codegreen},
    keywordstyle=\color{magenta},
    numberstyle=\tiny\color{codegray},
    stringstyle=\color{codepurple},
    basicstyle=\footnotesize,
    breakatwhitespace=false,         
    breaklines=true,                 
    captionpos=b,                    
    keepspaces=true,                 
    numbers=left,                    
    numbersep=5pt,                  
    showspaces=false,                
    showstringspaces=false,
    showtabs=false,                  
    tabsize=2
}

\lstset{escapechar=@,style=customc}

\begin{document}
	\thispagestyle{empty}
	\begin{center}
		\vspace*{1.6cm}
		\includegraphics[width=0.55\linewidth]{pblogo.png}	\\
		\vspace{0.5cm}
		\large
		\textsf{\textbf{Sprawozdanie z pracowni specjalistycznej}} \\
		\vspace{0.5cm}
		\textsf{\textbf{\textit{Systemy operacyjne}}}	\\
		\vspace{1cm}
		\textsf{Projekt numer: \textbf{\numerzajec}}	\\
		\vspace{0.5cm}
		\textsf{Temat: \textbf{\tematzajec}}
	\end{center}

	\vspace{2cm}
	\begin{tabular}{rl}
        Wykonujący ćwiczenie: &\textbf{Karol Budlewski} \\
                              & \textbf{Maciej Więckowski} \\
	\end{tabular}

	\vspace{3.5cm}	

	\begin{minipage}{0.45\linewidth}
		\large
		\begin{spacing}{1.5}
		Studia dzienne \\
		Kierunek: Informatyka \\
		Semestr: IV \\
		\end{spacing}
	\end{minipage}
	\begin{minipage}[t]{0.5\linewidth}
		\large
		\begin{spacing}{1.2}
		Grupa zajęciowa: PS4
		\end{spacing}
	\end{minipage}
	
	Prowadzący ćwiczenie: \textbf{mgr inż. Daniel Reska} \\
	
	\begin{flushright}
		\begin{minipage}[t]{0.3\linewidth}
			\centering
			................. \\
			\small OCENA
		\end{minipage}
	\end{flushright}
	
	\begin{minipage}[t]{0.4\linewidth}
		\centering
		Data \\
		\small \datazajec
	\end{minipage}
	
	\begin{flushright}
		\begin{minipage}[t]{0.5\linewidth}
			\centering
			............................................. \\
			\small \textsf{Data i podpis prowadzącego}
		\end{minipage}
	\end{flushright}
	\pagebreak
	%%%%%%%%%%%%%%%%%%%%%%%%%%%%%%%%%%%%%%%%%%%%%%%%%
	
	\section{Wstęp}
	Celem projektu było rozwiązanie problemu fryzjera realizowanego na wątkach oraz 
	synchronizowanego semaforami, mutexami i zmiennymi warunkowymi.\\
	Program zrealizowany jest w wariantach:
	\begin{enumerate}[label=\alph*)]
		\item z użyciem wyłącznie semaforów i mutexów.
		\item z użyciem zmiennych warunkowych.
	\end{enumerate}
	Oczekiwana liczba punktów to \texttt{34}.
	
	\section{Instalacja}
	Program jest instalowany z linii komend. Należy przejść do katalogu z plikiem 
	\texttt{Makefile} i uruchomić komendę \texttt{make}. Po tym mamy już skompilowany 
	program, który mieści się w katalogu \texttt{build}.
	
	\section{Sposób uruchamiania}
	Program jest uruchamiany z linii komend i może przyjmować następujące parametry:
	\begin{enumerate}[label=\alph*)]
		\item \texttt{-t}  - określa czas w \texttt{mikrosekundach} z jakim będą 
				się pojawiać nowe wątki \texttt{Klient}
		\item \texttt{-ts} - określa czas w \texttt{sekundach} z jakim będą 
				się pojawiać nowe wątki \texttt{Klient}
		\item \texttt{-tms} - określa czas w \texttt{milisekundach} z jakim będą 
				się pojawiać nowe wątki \texttt{Klient}
		\item \texttt{-size} - określa wielkość poczekalni
		\item \texttt{-debug} - uruchamia tryb \texttt{debug}, wypisujący całą 
				kolejkę klientów czekających oraz tych którzy nie dostali się do 
				gabinetu 
	\end{enumerate}
	Do uruchomienia programu nie wymagane są żadne parametry. Program uruchomi 
	się wówczas z ustawieniami domyślnymi, czyli czas ustawiony na 75 ms, rozmiar 
	poczekalni na 10 oraz bez debuga. Parametry \texttt{-t}, \texttt{-ts}, \texttt{-tms} 
	wpływają na siebie i zawsze będzie brany ostatni pod uwagę oraz jeżeli zostanie 
	popełniony błąd w którym kolwiek z opcji program nie zostanie uruchomiony.
	\subsubsection*{Przykłady uruchamiania programu}
	\begin{itemize}
		\item ./build/fryzjer -t   150
		\item ./build/fryzjer -ts  150
		\item ./build/fryzjer -tms 150
		\item ./build/fryzjer -ts  150 -debug
		\item ./build/fryzjer -tms 150 -size 10
		\item ./build/fryzjer -t   150 -size 10 -debug
	\end{itemize}
 	 	
 	\section{Zaimplementowane funkcje}
 	\subsection{Ogólne założenia funkcji}
 	W obu programach występują te same funkcje:
 	\begin{enumerate}
 		\item \texttt{customer}, 
 		\item \texttt{barber}, 
 		\item \texttt{parser}, 
 		\item \texttt{createCustomer}, 
 		\item \texttt{state}, 
 		\item \texttt{semaphoreInit}.
	\end{enumerate}
 	Funckje \texttt{parser}, \texttt{createCustomer}, \texttt{state}, 
 	\texttt{semaphoreInit} wykonują ten sam kod. Dodatkowo została zaimplementowana 
 	kolejka \textsc{FIFO}, która jest używana również jak zwykła lista.
 	\subsubsection{parser}
 	Jest to funkcja do sprawdzania użytych opcji. Może spowodować wykonanie 
 	się programu na domyślnych ustawieniach w przypadku błędnego użycia opcja.
 	
 	np. po opcji \texttt{-size} nie zostanie podana żadna liczba.
 	
 	\subsubsection{state}
 	Jest to funkcja do wyświetlania na ekranie aktualnego stanu salonu fryzjerskiego. 
 	Gdy w programie jest włączona opcja \texttt{debug} to ta funkcja powoduje również 
 	wypisanie na ekranie listy klientów, którzy zrezygnowali oraz przebywających w 
 	poczekalni.
 	
 	\subsubsection{semaphoreInit}
 	Jest to funkcja do inicjalizacji semaforów.
 	
 	\subsubsection{createCustomer}
 	Jest to funkcja do stworzenia wątku \texttt{Klienta}.\\\\
 	Pozostałe funkcje się różnią od wersji programu i ze względu na użycie synchronizacji 
 	zostano szczegółowo przedstawione.
 	\pagebreak
 	\subsection{Wersja bez użycia zmiennych warunkowych}
 	\subsubsection{Wątek klienta}
 	\begin{lstlisting}
	void customer(void* arg) {
			// Zablokowanie mutexu odpowiedzialnego za wydarzeniami w poczekalni
			// Klient wszedl do poczekalni
    	pthread_mutex_lock(&mutex_poczekalnia);
    	// Klient pobiera numerek
    	int id = licznik++;
    	// Klient sprawdza czy jest wystarczajaca ilosc miejsc w poczekalni
    	if(sem_trywait(&poczekalnia) == 0){
    			// Klientowi udalo sie wejsc do poczekalni i zajac miejsce
        	enqueue(oczekujacy_klienci, id, NULL);
        	state();
        	// Klient sygnalizuje fryzjerowi ze jest gotowy do strzyzenia
        	// poprzez wyslanie do niego sygnalu
        	sem_post(&budzenie);
        	// Klient zostal zaproszony do strzyzenia
        	// Mutex poczekalni zostal zwolniony
        	pthread_mutex_unlock(&mutex_poczekalnia);
			
					// Klient wszedl do gabinetu fryzjera
					// Mutex gabinetu zostaje zablokowany
    	    pthread_mutex_lock(&mutex_gabinet);
					// Klient idzie na strzyzenie
        	sem_wait(&strzyzenie_klienta);
        	// Zwolnione zostaje miejsce w poczekalni
	        sem_post(&poczekalnia);
	        klient = id;
        	oczekujacy_klienci = dequeue(oczekujacy_klienci);
	        state();
        	usleep(100000);
        	// Klient zostaje ostrzyzony
    	    sem_post(&zajetosc_fryzjera);
	        klient = -1;
        	state();
        	// Klient wychodzi od fryzjera
        	// Mutex gabinetu zostaje odblokowany
        	pthread_mutex_unlock(&mutex_gabinet);
    	}
    	else {
    			// Klientowi nie udalo sie zajac miejsca wiec zrezygnowal z wizyty 
    			// u fryzjera
        	++zrezygnowani;
	        enqueue(zrezygnowani_klienci, id, NULL);
    	    state();
    	    // Mutex poczekalni zostal zwolniony po wyjsciu Klienta
   		    pthread_mutex_unlock(&mutex_poczekalnia);
	    }
	}
	\end{lstlisting}
	\pagebreak
	\subsubsection{Wątek fryzjera}
	\begin{lstlisting}
	void barber(void* arg) {
			// Fryzjer pracuje caly czas
    	while(1) {
    			// Fryzjer zostaje powiadomiony  o tym ze jest klient
        	sem_wait(&budzenie);
        	// Fryzjer bierze Klienta z poczekalni
        	// Mutex poczekalni zostaje zablokowny
	        pthread_mutex_lock(&mutex_poczekalnia);
	        // Klient zostaje ostrzyzony
     	   	sem_post(&strzyzenie_klienta);
     	   	// Fryzjer wypuszcza Klienta
     	   	// Mutex zostaje odblokowny
      	  pthread_mutex_unlock(&mutex_poczekalnia);
      	  // Fryzjer mowi ze jest wolny
     	   	sem_wait(&zajetosc_fryzjera);
        	usleep(100000);
	    }
	}
	\end{lstlisting}
	
 	\subsection{Wersja z użyciem zmiennych warunkowych}
 	\subsubsection{Wątek fryzjera}
	\begin{lstlisting}
	void barber(void* arg) {
			// Fryzjer pracuje caly czas
    	while(1) {
    			// Fryzjer zostaje powiadomiony  o tym ze jest klient
        	sem_wait(&budzenie);
        	// Fryzjer bierze Klienta z poczekalni
        	// Mutex poczekalni zostaje zablokowny
	        pthread_mutex_lock(&mutex_poczekalnia);
					pthread_cond_t *zmienna = oczekujacy_klienci->next->zmienna;
					// Klient zostaje ostrzyzony
					// Wysyla sygnal przez zmienna warunkowa
        	pthread_cond_signal(zmienna);     	   	
        	// Fryzjer wypuszcza Klienta
     	   	// Mutex zostaje odblokowny
      	  pthread_mutex_unlock(&mutex_poczekalnia);
      	  // Fryzjer mowi ze jest wolny
     	   	sem_wait(&zajetosc_fryzjera);
        	usleep(100000);
	    }
	}
	\end{lstlisting}
	\pagebreak
 	\subsubsection{Wątek klienta}
 	\begin{lstlisting}
	void customer(void* arg) {
			// Zablokowanie mutexu odpowiedzialnego za wydarzeniami w poczekalni
			// Klient wszedl do poczekalni
    	pthread_mutex_lock(&mutex_poczekalnia);
    	// Klient pobiera numerek
    	int id = licznik++;
    	// Przypisanie zmiennej warunkowej do klienta
    	pthread_cond_t *cond_klient = malloc(sizeof(pthread_cond_t));
    	pthread_cond_init(cond_klient, NULL);
    	// Klient sprawdza czy jest wystarczajaca ilosc miejsc w poczekalni
    	if(sem_trywait(&poczekalnia) == 0){
    			// Klientowi udalo sie wejsc do poczekalni i zajac miejsce
        	enqueue(oczekujacy_klienci, id, cond_klient);
        	state();
        	// Klient sygnalizuje fryzjerowi ze jest gotowy do strzyzenia
        	// poprzez wyslanie do niego sygnalu
        	sem_post(&budzenie);
        	// Klient czeka na zaproszenie do strzyzenia
        	pthread_cond_wait(cond_klient, &mutex_poczekalnia);
        	// Zwolnione zostaje miejsce w poczekalni
        	sem_post(&poczekalnia);
        	// Klient zostal zaproszony do strzyzenia
        	// Mutex poczekalni zostal zwolniony
        	pthread_mutex_unlock(&mutex_poczekalnia);
			
					// Klient wszedl do gabinetu fryzjera
					// Mutex gabinetu zostaje zablokowany
    	    pthread_mutex_lock(&mutex_gabinet);
	        klient = id;
        	oczekujacy_klienci = dequeue(oczekujacy_klienci);
	        state();
        	usleep(100000);
        	// Klient zostaje ostrzyzony
    	    sem_post(&zajetosc_fryzjera);
	        klient = -1;
        	state();
        	// Klient wychodzi od fryzjera
        	// Mutex gabinetu zostaje odblokowany
        	pthread_mutex_unlock(&mutex_gabinet);
    	}
    	else {
    			// Klientowi nie udalo sie zajac miejsca wiec zrezygnowal z wizyty 
    			// u fryzjera
        	++zrezygnowani;
	        enqueue(zrezygnowani_klienci, id, NULL);
    	    state();
    	    // Mutex poczekalni zostal zwolniony po wyjsciu Klienta
   		    pthread_mutex_unlock(&mutex_poczekalnia);
	    }
	}
	\end{lstlisting}

 	\section{Sposób odinstalowania programu}
 	Demon jest odinstalowany z linii komend. Należy przejść do katalogu z plikiem 
 	\texttt{Makefile} i uruchomić komendę \texttt{make clean}.

\end{document}
